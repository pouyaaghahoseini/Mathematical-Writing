\documentclass{article}
\usepackage{fullpage,graphicx}
\usepackage{amsmath,amsfonts,amsthm,amssymb,multirow,xcolor}

\begin{document}
\noindent
Mathematical Writing \hfill \textbf{Exercise III} \newline 
{5-April-2019} \hfill Pouya Aghahoseini

\noindent
\rule{\linewidth}{0.4pt}
\textbf{\large\color{blue} Exercise 2.5}   \textbf{Turn symbols into words}
	\begin{enumerate}
		\item 
		The inverse of a function at zero.
		\item
		The reciprocal of the value of a function at zero.
		\item 
		The inverse of the composition of a function with another function.
		\item 
		The image of positive real numbers under a function.
		\item 
		The intersection of the image of A under $f$ and the image of B under $f$.
		\item 
		The intersection of the integer set(Z) with its inverse image under a function.
		\item 
		The intersection of the rational numbers with the image of real numbers under a function.
	\end{enumerate}
	
\text{}\\
\textbf{\large\color{blue} Exercise 2.2}   \textbf{Explain Clearly and Plainly.}
	\begin{enumerate}
		\item 
		Divide your number by integer numbers less than it's square root, if it isn't dividable by any of them then it's a prime number.
		\item 
		Subtract square numbers such as 1,4,9,... from your number, if the result is a square number then your number is sum of two square numbers.
		\item 
		All three parameters of lines must be equal at the point of intersection, this gives us three equations with two unknowns $(t,t')$,the result of these three equations must be compatible with each other in order to have an intersection.
		\item
		The perpendicular bisectors of two chords meet at the center of the circle. so we find the line equations of two chords, then we find their perpendicular bisectors equation, their intersection point is the center of the circle.
		\item 
		If the coefficient of the term $x^{2}$ is positive(negative) then the function will definitely assume positive(negative) values somewhere on the graph. all we need to do is whether it assumes negative(positive) values where it's derivative is zero, so you have to check the sign of $f(\frac{-b}{2a})$. 
	\end{enumerate}
\end{document}
