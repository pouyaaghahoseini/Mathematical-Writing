\documentclass{article}
\usepackage{fullpage,graphicx}
\usepackage{amsmath,amsfonts,amsthm,amssymb,multirow,xcolor}
\usepackage{algorithmic}
\usepackage[ruled,vlined,commentsnumbered,titlenotnumbered]{algorithm2e}

\begin{document}
\noindent
Mathematical Writing \hfill \textbf{Exercise II} \newline 
{4/15/2019} \hfill Pouya Aghahoseini

\noindent
\rule{\linewidth}{0.4pt}
\textbf{\large\color{blue} Exercise 2.1}
\begin{itemize}
	\item \textbf{Prime Numbers}
	\begin{enumerate}
		\item 
		There are 25 prime numbers less than 100.
		\item
		The only even prime number is 2.
		\item 
		No prime number greater than 5 ends with a 5.
		\item 
		The set of prime numbers is infinite.
		\item 
		The largest known prime number as of now is $2^{82,589,933}$ with 24,862,048 digits.
		\item 
		How do you know if a number is prime?
		\item 
		Why is it important to find the largest prime number?
		\item 
		How do you prove a number is prime?
		\item 
		Is there a largest prime number?
		\item 
		Why is 1 not a prime number?
	\end{enumerate}
	\item \textbf{Fractions}
	\begin{enumerate}
		\item 
		$\frac{9/10}{den}$ is greater than $\frac{8}{9}$.
		\item 
		An improper fraction is always 1 or greater than 1.
		\item 
		 A fraction is a numerical quantity that is not a whole number.
		\item 
		 A fraction is proper if the numerator is less than the denominator.
		\item 
		Any integer number can be shown as a fraction.
		\item 
		What is the difference between a fraction and a rational number?
		\item 
		What are the equivalent fractions of $\frac{2}{3}$?
		\item 
		How to solve fractions?
		\item 
		Are fractions always less than 1?
		\item
		How can you tell which fraction is greater?
	\end{enumerate}
	\item \textbf{Complex Numbers}
	\begin{enumerate}
		\item 
		The 16th century Italian mathematician Gerolamo Cardano introduced complex numbers in his attempts to find solutions to cubic equations.
		\item 
		The complex numbers cannot be ordered since the square of the imaginary unit i is −1.
		\item
		You can't compare two complex numbers in some cases.
		\item 
		The product of two imaginary numbers is always a negative real number.
		\item 
		The set of integers is closed under addition, multiplication, and exponentiation, but not division.
		\item 
		Is $2i$ a complex number?
		\item 
		Can complex numbers be ordered?
		\item 
		Are all numbers complex?
		\item 
		Can we compare two complex numbers?
		\item 
		What is the real part of $i^{i+5}?$
	\end{enumerate}
\end{itemize}
\text{}\\
\textbf{\large\color{blue} Exercise 2.2}
	\begin{enumerate}
		\item 
		The set of the trivial subsets of a finite set.
		\item 
		The set of solutions of a quadratic equation.
		\item 
		The set of moons around planet mars.
		\item
		The set of drops of water in ocean.
		\item 
		The set of sand grains on land.
	\end{enumerate}
\textbf{\large\color{blue} Exercise 2.3}
	\begin{enumerate}
		\item 
		$\{2z+1 : z \in Z^{-}\}$
		\item 
		$\{n \in N : 99<n<1000\}$
		\item 
		$\{\frac{x+1}{x} : x \in Z^{*} \}$
		\item 
		$\{(x,y,z) \in Q^{3} : x^{2}+y^{2}+z^{2} \leq 1 \}$
		\item 
		$\{z \in C : |z| \geq 1\}$
		\item 
		$\{(x,y,z) \in R^{3} : x^{2}+y^{2}+z^{2}=1 \}$
		\item 
		$\{(x-a)^{2}+(y-b)^{2}=R^{2} : a^{2}+b^{2}=R^{2} \}$
		\item 
		$\{y=\frac{k}{x} : k,x,y \in R \}$
		\item 
		$\{ ax+by+1=0 : \text{a,b values result in tangency.} \}$
	\end{enumerate}
\textbf{\large\color{blue} Exercise 2.4}
	\begin{enumerate}
		\item 
		The set of rational points in the open unit interval.
		\item 
		The set of reciprocals of odd integers.
		\item 
		The set of rationals whose numerator is odd and its denominator is a power of 2.
		\item 
		The set of real roots of integer numbers excluding integers.
		\item 
		The imaginary axis in the complex plane, excluding the origin.
		\item 
		The set of complex numbers where sum of real and imaginary parts is less than 1.
		\item 
		The set of integer pairs whose first component divides the second.
		\item 
		The set of vectors in the three-dimensional space where at least one coordinate is zero.
		\item 
		The set of points in a euclidean space whose coordinates have zero sum.
		\item 
		The set of all integers divided by 2.
	\end{enumerate}
\end{document}
