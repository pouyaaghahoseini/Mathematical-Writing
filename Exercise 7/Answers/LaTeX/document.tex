\documentclass{article}
\usepackage{fullpage,graphicx}
\usepackage{amsmath,amsfonts,amsthm,amssymb,multirow,xcolor}
\usepackage{enumitem}
\begin{document}
	\noindent
	Mathematical Writing \hfill \textbf{Exercise VII} \newline 
	{17-Jun-2019} \hfill Pouya Aghahoseini
	
	\noindent
	\rule{\linewidth}{0.4pt}
	\textbf{\large\color{blue} Exercise 6.1}   \textbf{Answer concisely. }
	
	\begin{enumerate}
		\item 
		\textbf{What is the difference between an ordered pair and a set?}\\
		An ordered pair always has two elements, while a set can contain any arbitrary amount of elements without any constraints on order.
		\item 
		\textbf{What is the difference between an equation and an identity?}\\
		The equation is true for particular values of the variable(or variables). But an identity is true for all values. 		
		\item 
		\textbf{What is the difference between a function and its graph?}\\
		The graph of a function f is the set of all points in the plane of the form (x, f(x)), but a function is a set ordered pairs.
	\end{enumerate}
	
	\text{}\\
	\textbf{\large\color{blue} Exercise 6.3} \textbf{Consider the following question:}\\
	\textit{I drive ten miles at 30 miles an hour, and then another ten
		miles at 50 miles an hour. It seems to me my average speed
		over the journey should be 40 miles an hour, but it doesn't
		work out that way. Why not?}\\
	
	\textbf{Write an explanation for the general public, clarifying why
		such a confusion may arise.You may perform some basic
		arithmetic, but do not use symbols as most people find them
		difficult to understand.}\\
	
		As it was mentioned, you drove 20 miles in some time, which in order to estimate your average speed, you need to divide the distance to the time it has taken. Instead, you have evaluated the average of your speeds.
This is true only in the case that you do not have any acceleration but since you have an acceleration average speed is $\frac{20}{1/3+1/5}$ .

	
	
	\text{}\\
	\textbf{\large\color{blue} Exercise 6.4} \textbf{Consider the following question:}\\
	\textit{I tossed a coin four times, and got heads four times.
		It seems to me that if I toss it again I am much
		more likely to get tails than heads, but it doesn't
		work out that way. Why not?}\\
	
	\textbf{Write an explanation for the general public,
		clarifying why such a confusion may arise. You may
		use symbols such as H and T for heads-tails
		outcomes, but avoid using other symbols.}\\
	
		You have to note that the fifth coin toss is completely distinct from previous coin tosses. Actually, all coin tosses are distinct from each other because they don't affect each other in any manner. so in order to evaluate the probability of the fifth coin to be Tails is equal to Heads and is 1/2.
\end{document}