\documentclass{article}
\usepackage{fullpage,graphicx}
\usepackage{amsmath,amsfonts,amsthm,amssymb,multirow,xcolor}
\usepackage{enumitem}
\begin{document}
\noindent
Mathematical Writing \hfill \textbf{Excercise IV} \newline 
{12-April-2019} \hfill Pouya Aghahoseini

\noindent
\rule{\linewidth}{0.4pt}
\textbf{\large\color{blue} Exercise 3.2}   \textbf{For each expression, provide two levels of description:}

	\begin{enumerate}
		\item 
		\begin{enumerate}[label=\roman*]
			\item  An identity.
			\item An arithmetical identity, expressing a cube as the sum of three cubes.
		\end{enumerate}
	\item
		\begin{enumerate}[label=\roman*]
			\item  An arithmetical expression.
			\item A quadratic surd, with same radicands.
		\end{enumerate}
	\item
		\begin{enumerate}[label=\roman*]
			\item  A chain of inequalities.
			\item Upper and lower rational bounds for the square root of 2.
		\end{enumerate}
	\item
		\begin{enumerate}[label=\roman*]
			\item  An algebraic expression.
			\item  An algebraic expression consisting of three monomials.
		\end{enumerate}
	\item
		\begin{enumerate}[label=\roman*]
			\item  An inequality.
			\item An algebraic inequality in two unknowns.
		\end{enumerate}
		\item
		\begin{enumerate}[label=\roman*]
			\item  An identity.
			\item An algebraic formmula for the expansion of the cube of an expression.
		\end{enumerate}
		\item
		\begin{enumerate}[label=\roman*]
			\item  An equality.
			\item The cartesian equation of a parabola passing through the origin.
		\end{enumerate}
		\item
		\begin{enumerate}[label=\roman*]
			\item  An equation.
			\item  An equation which has no answers. (The solution set is empty.)
		\end{enumerate}
		\item
		\begin{enumerate}[label=\roman*]
			\item  An identity.
			\item The trigonometric formula for the sine of the difference of two angles.
		\end{enumerate}
		\item 
		\begin{enumerate}[label=\roman*]
			\item  An equation
			\item A differential equation, with the underlying ambient set of differentiable functions.
		\end{enumerate}
		\item
		\begin{enumerate}[label=\roman*]
			\item  An inequality.
			\item  An inequality with multivariate functions on each side.
		\end{enumerate}
		\item
		\begin{enumerate}[label=\roman*]
			\item  A system of equations.
			\item  A system of two simultaneous equations in 2 unknowns.
		\end{enumerate}
		\item
		\begin{enumerate}[label=\roman*]
			\item  An identity.
			\item  An identity, expressing associative law on sets.
		\end{enumerate}		
			\item
		\begin{enumerate}[label=\roman*]
			\item  An identity.
			\item  A formula for the infinite summation of reciprocal of the fourth power of natural numbers.
		\end{enumerate}		
	\end{enumerate}
	
\text{}\\
\textbf{\large\color{blue} Exercise 3.4}

\begin{enumerate}
	\item 
	\begin{enumerate}[label=\roman*]
		\item  A function.
		\item The real function that adds 1 to its argument.
	\end{enumerate}
	\item 
\begin{enumerate}[label=\roman*]
	\item  A function.
	\item The integral of a rational function.
\end{enumerate}
	\item 
\begin{enumerate}[label=\roman*]
	\item  An identity. (A functional identity.)
	\item  The formula for the derivative of the product of two functions.
\end{enumerate}
	\item 
\begin{enumerate}[label=\roman*]
	\item  An identity. (A functional identity.)
	\item  The formula for integral of a function with substituted unknown.
\end{enumerate}
	\item 
\begin{enumerate}[label=\roman*]
	\item  An integral.
	\item  The indefinite integral of a function of two variables, performed with respect to the first variable.
\end{enumerate}
	\item 
\begin{enumerate}[label=\roman*]
	\item  An integral.
	\item  The indefinite double integral of a function, performed with respect to the xy plane.
\end{enumerate}
	\item 
\begin{enumerate}[label=\roman*]
	\item  An identity. (A definition.)
	\item  The power series of the cosine.
\end{enumerate}
	\item 
\begin{enumerate}[label=\roman*]
	\item  A derivative.
	\item  The sum of partial derivatives of a multivariate function.
\end{enumerate}
	\item 
\begin{enumerate}[label=\roman*]
	\item  A finite product of functions.
	\item  The product of all the partial derivatives of a function of several variables.
\end{enumerate}
	\item 
\begin{enumerate}[label=\roman*]
	\item  An infinite summation.
	\item  Infinite summation of an unknown raised to square numbers.
\end{enumerate}
	\item 
\begin{enumerate}[label=\roman*]
	\item  An integral
	\item  The infinite integral of a function containing napier's constant.
\end{enumerate}
	\item 
\begin{enumerate}[label=\roman*]
	\item  An infinite product
	\item  The infinite product of a function divided by square numbers.
\end{enumerate}
\end{enumerate}
\text{}\\
\textbf{\large\color{blue} Exercise 3.5}

\begin{enumerate}
	\item 
	\begin{enumerate}[label=\roman*]
		\item A Set.
		\item The intersection of the inverse images of the elements of a sequence of sets.
	\end{enumerate}
	\item 
\begin{enumerate}[label=\roman*]
	\item A Number.
	\item The size of the infinite union of a sequence of power sets.
\end{enumerate}
	\item 
\begin{enumerate}[label=\roman*]
	\item A Set equation.
	\item The set equation expressing the union of sets over a function equals union of each individual set over the function.
\end{enumerate}
	\item 
\begin{enumerate}[label=\roman*]
	\item A Set.
	\item The set is cartesian product of n copies of irrational multiples of integer set. 
\end{enumerate}
	\item 
\begin{enumerate}[label=\roman*]
	\item A Set.
	\item The set of all infinite sequences of the set of non-negative and less than 2 powers of x.
\end{enumerate}
	\item 
\begin{enumerate}[label=\roman*]
	\item A Set.
	\item The set of doubly-infinite sequences of elements of Z.
\end{enumerate}
	\item 
\begin{enumerate}[label=\roman*]
	\item An Equation.
	\item A set equation with and empty solution set.
\end{enumerate}
	\item 
\begin{enumerate}[label=\roman*]
	\item An Equation.
	\item The set of functions which multiply previous argument and its value in function as the result.
\end{enumerate}
	\item 
\begin{enumerate}[label=\roman*]
	\item An Equation
	\item A functional equation where composition of the function for n times results the identity function.
\end{enumerate}
	\item 
\begin{enumerate}[label=\roman*]
	\item A Set.
	\item The Minkowski sum of an interval for n times. (The closed interval of $[0,n]$.)
\end{enumerate}
\end{enumerate}\end{document}